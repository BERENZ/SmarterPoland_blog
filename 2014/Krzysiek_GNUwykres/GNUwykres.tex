\documentclass[a4paper,titlepage,12pt]{mwart}
\usepackage[left=3cm, right=2.5cm, top=2.5cm, bottom=2.5cm]{geometry} % Boltzman
\usepackage{latexsym}
\usepackage[dvips]{graphicx}
\usepackage{tabularx}
\usepackage{booktabs}

\usepackage{gnuplottex}
\usepackage{pst-slpe}
\usepackage{pst-text}
\usepackage{pst-grad}
\usepackage{pstcol}
\usepackage{filecontents}
\usepackage{pst-bspline}

\usepackage{pst-math}
\usepackage{pstricks-add}
\usepackage{pst-func}
\usepackage{pst-infixplot}

\usepackage{psfrag}   % podmiana czcionek w GNUplot
\usepackage{fancyhdr} % nagłówek i stopka
\usepackage{ucs}			% polskie czcionki w komentarzach listingu dla utf-8
\usepackage{textcomp} % dotyczy EURO
\usepackage{color}   % kolor czcionek
\usepackage{array}      % kolory tabeli
\usepackage{hhline}     % kolory tabeli
\usepackage{dcolumn}    % kolory tabeli
\usepackage{longtable}  % kolory tabeli
\usepackage{colortbl}   % kolory kolumn, wierszy i pojedyńczych cel tabel
\usepackage[footnotesize,labelsep = period,labelfont={bf},textfont={rm}]{caption}

\usepackage{polski}   % polskie znaki
\usepackage[OT4]{fontenc}    % polskie znaki
\usepackage[utf8]{inputenc}  % polskie znaki
\usepackage{amsmath}			% numeracja wzorów zgodnie rozdziałami
\numberwithin{equation}{section}	% numeracja wzorów zgodnie rozdziałami
\numberwithin{table}{section}           % numeracja tabel zgodnie z rozdziałami
\numberwithin{figure}{section}          % numeracja rysunków zgodnie z rozdziałami

\newtheorem{tw}{Twierdzenie}[section]          % dotyczy twierdzeń
\newtheorem{lem}[tw]{Lemat}                    % dotyczy Lematu
\newtheorem{defi}{Definicja}[section]          % dotyczy definicji
\newtheorem{prz}{\sc{Przykład}}[section]       % dotyczy przykładu
\usepackage{supertabular}
\usepackage{rotating}                   % obrót tabeli lub rysunku
\usepackage{dcolumn}
\usepackage{multicol}
\usepackage{listings}
\usepackage{subfig}

\usepackage{multirow}        % łączenie równań tabeli
\sloppy 
\usepackage{hyperref}
\usepackage{url}

\renewcommand\captionlabelfont{\bfseries}
\newcommand {\otoprule }{\midrule [\heavyrulewidth]}

\title{\bf{Gnuplot}}

\author{Krzysztof Trajkowski}
\date{\today}

\lstset{backgroundcolor=\color{yellow!10},
rulecolor=\color{gray!70},
language=Gnuplot,frame=lines,numbers=left,numberstyle=\tiny,numbersep=5pt,
breaklines=true,
numbers=none,
basicstyle=\footnotesize\ttfamily,
commentstyle=\footnotesize\ttfamily\itshape\color{red},captionpos=t,
prebreak = \raisebox{0ex}[0ex][0ex]{\ensuremath{\hookleftarrow}},
inputencoding=utf8x,
    extendedchars=\true,
    literate={ą}{{\k{a}}}1
             {Ą}{{\k{A}}}1
             {ę}{{\k{e}}}1
             {Ę}{{\k{E}}}1
             {ó}{{\'o}}1
             {Ó}{{\'O}}1
             {ś}{{\'s}}1
             {Ś}{{\'S}}1
             {ł}{{\l{}}}1
             {Ł}{{\L{}}}1
             {ż}{{\.z}}1
             {Ż}{{\.Z}}1
             {ź}{{\'z}}1
             {Ź}{{\'Z}}1
             {ć}{{\'c}}1
             {Ć}{{\'C}}1
             {ń}{{\'n}}1
             {Ń}{{\'N}}1
}

\begin{document}
\maketitle
\section*{Wprowadzenie}
Gnuplot jest wolnym, wieloplatformowym oprogramowaniem do rysowania wykresów matematycznych oraz do graficznej prezentacji już obrobionych danych np. zapisanych w pliku. Ta aplikacja ma tylko kilka wbudowanych funkcji do obróbki danych np. krzywe Beziera, interpolacja (natural cubic splines algorithm), aproksymacja tzw. fitowanie (Marquardt-Levenberg algorithm). Warto też wspomnieć, że za pomocą funkcji {\verb stats } możemy obliczyć podstawowe statystyki tj. min., max., średnia, kwartyle.
Wiele aplikacji wykorzystuje gnuplota do grafki np. Octave -- pakiet do obliczeń numerycznych, Maxima -- pakiet do obliczeń symbolicznych, Gretl -- pakiet do estymacji modeli ekonometrycznych. Warto podkreślić, że  Gretl daje możliwość zapisu wykresu do pliku w formie poleceń gnuplota. Dzięki tej opcji możemy dowolnie modyfikować parametry wykresów. Dodatkowo w Gretlu możemy pisać i uruchamiać pliki skryptowe gnuplota. Warto więc poznać język komend tego programu graficznego.

\section{Gnuplot a \LaTeX}
Gnuplot ma wiele formatów wyjściowych np. PostScript, GIF, EMF, SVG, PDF lub PNG. Aby zobaczyć wszystkie jakie są dostępne wystarczy w konsoli programu wpisać poniższą komendę.
\begin{lstlisting}
set terminal
\end{lstlisting}
Więcej informacji o ustawieniach danego terminala można znaleźć w dokumentacji
\begin{lstlisting}
help terminal pdf
\end{lstlisting}
Wiele graficznych możliwości wybranego wcześniej terminala możemy zobaczyć na rysunku, wpisując w konsoli poniższą komendę.
\begin{lstlisting}
set terminal pdf
test
\end{lstlisting}
\begin{figure}[!ht]
\begin{center}
\begin{gnuplot}[terminal=pdf,terminaloptions={enhanced size 12cm,7cm}]
test
\end{gnuplot}
\end{center}
\caption{Terminal PDF.}
\end{figure}

Dla osób składających swoje dokumenty w \LaTeX-u
ciekawym rozwiązaniem jest użycie pakietu {\verb gnuplottex } dzięki któremu możemy umieszczać kod gnuplota w pliku \LaTeX-a. 
\begin{lstlisting}
\begin{figure}[!ht]
\begin{center}
\begin{gnuplot}[terminal=pdf,terminaloptions={enhanced font "Times-Roman,10" size 10cm,6cm}]
# tutaj wpisujemy kod gnuplota:
set xrange[-4:4]
set xlabel "x"
set ylabel "y"
set label "{/Symbol m} = 0, {/Symbol s}^2 = 1" at -3.5, 0.35 left
set label "{/Symbol \362@_{/=12 - \245}^{/=8 + \245}} f(x)= 1" at 1.5, 0.35 center
Gauss(x,mu,sigma) = 1./(sigma*sqrt(2*pi)) * exp( -(x-mu)**2 / (2*sigma**2) )
plot Gauss(x,0,1) t "" lc rgb "orange" lw 5
\end{gnuplot}
\end{center}
\end{figure}
\end{lstlisting}

\begin{figure}[!ht]
\begin{center}
%\begin{scriptsize}
\begin{gnuplot}[terminal=pdf,terminaloptions={enhanced font "Times-Roman,10" size 12cm,7cm}]
set xrange[-4:4]
set xlabel "x"
set ylabel "y"
set label "{/Symbol m} = 0, {/Symbol s}^2 = 1" at -3.5, 0.35 left
set label "{/Symbol \362@_{/=12 -\245}^{/=8 +\245}} f(x)= 1" at 1.5, 0.35 center
Gauss(x,mu,sigma) = 1./(sigma*sqrt(2*pi)) * exp( -(x-mu)**2 / (2*sigma**2) )
plot Gauss(x,0,1) t "" lc rgb "orange" lw 5
\end{gnuplot}
%\end{scriptsize}
\end{center}
\caption{Standardowy rozkład normalny -- terminal PDF.}
\label{Gpdf}
\end{figure}

Jeśli na wykresie chcemy umieszczać wzory matematyczne lub symbole greckie używanie terminala PDF lub EPS nie jest optymalnym rozwiązaniem -- Rysunek \ref{Gpdf}. Jednym z rozwiązań tego problemu może być modyfikacja wykresu tzn. podmiana ciągu znaków (np. symbole matematyczne) w pliku EPS na \LaTeX-owe. Można w tym celu wykorzystać pakiet {\verb psfrag }. Jednak lepszym rozwiązaniem jest wykorzystanie terminala {\verb epslatex }. Jest to dobra alternatywa dla zestawu makr PSTricks.

\begin{lstlisting}
set terminal epslatex
test
\end{lstlisting}
\newpage
\begin{figure}[!ht]
\begin{center}
\begin{scriptsize}
\begin{gnuplot}[terminal=epslatex,terminaloptions={font 8 color colortext size 12cm,7cm}]
test
\end{gnuplot}
\end{scriptsize}
\end{center}
\caption{Terminal epslatex.}
\end{figure}

\begin{lstlisting}
\begin{gnuplot}[scale=1,terminal=epslatex,terminaloptions={font 8 color colortext size 12cm,5.5cm}]
set xrange[-4:4]
set xlabel "x"
set ylabel "y"
set label "$\\mu=0,\\;\\sigma^2=1$" at -3.5, 0.35 left
set label "$\\int_{-\\infty}^{+\\infty}f(x)=1$" at 1.5, 0.35 left
Gauss(x,mu,sigma) = 1./(sigma*sqrt(2*pi)) * exp( -(x-mu)**2 / (2*sigma**2) )
plot Gauss(x,0,1) t "" lc rgb "orange" lw 5
\end{gnuplot}
\end{lstlisting}
\begin{figure}[!ht]
\begin{center}
\begin{scriptsize}
\begin{gnuplot}[scale=1,terminal=epslatex,terminaloptions={font 8 color colortext size 12cm,5.5cm}]
set xrange[-4:4]
set tmargin 0.5
set xlabel "x"
set ylabel "y"
set label "$\\mu=0,\\;\\sigma^2=1$" at -3.5, 0.35 left
set label "$\\int_{-\\infty}^{+\\infty}f(x)=1$" at 1.5, 0.35 left
Gauss(x,mu,sigma) = 1./(sigma*sqrt(2*pi)) * exp( -(x-mu)**2 / (2*sigma**2) )
plot Gauss(x,0,1) t "" lc rgb "orange" lw 5
\end{gnuplot}
\end{scriptsize}
\end{center}
\caption{Standardowy rozkład normalny -- terminal epslatex.}
\end{figure}

\section{Wykresy danych z pliku}
\subsection{Szereg czasowy}
Pierwszy przykład to prezentacja kształtowania się cen otwarcia indeksu giełdowego WIG20 oraz ich wygładzenie krzywą Beziera.

\begin{figure}[!ht]
\begin{center}
%\begin{footnotesize}
\begin{scriptsize}
\begin{gnuplot}[scale=1,terminal=epslatex,terminaloptions={font 8 color colortext size 15.5cm,7cm}]
set xdata time
set timefmt "%Y-%m-%d"
set format x "%Y\n%b"
set datafile separator ","
set xrange ["2011-12":"2013-12"]
set mxtics 3
set ylabel "Cena otwarcia"
set bmargin 0
set grid
plot "GNUwykres/DATA/wig20_d.csv" u 1:2 with lines title columnhead lc rgb "orange" lw 5, "GNUwykres/DATA/wig20_d.csv" u 1:2 smooth sbezier title "krzywa Bezier" lc rgb "black" lw 1 lt 1 
\end{gnuplot}
\end{scriptsize}
%\end{footnotesize}
\end{center}
\caption{Cena otwarcia indeksu WIG-20 w okresie od 01-01-2012 do 01-10-2013.}
\end{figure}
Lista komend gnuplota jaką wykorzystaliśmy do wygenerowania tego wykresu jest zamieszczona poniżej.

\begin{lstlisting}
set xdata time
set timefmt "%Y-%m-%d"
set format x "%Y\n%b"
set datafile separator ","
set xrange ["2011-12":"2013-12"]
set mxtics 3
set ylabel "Cena otwarcia"
set bmargin 0
set grid
plot "GNUwykres/DATA/wig20_d.csv" u 1:2 with lines title columnhead lc rgb "orange" lw 5, "GNUwykres/DATA/wig20_d.csv" u 1:2 smooth sbezier title "krzywa Bezier" lc rgb "black" lw 1 lt 1
\end{lstlisting}

Poniżej kolejny przykład wykresów giełdowych.
\begin{figure}[!ht]
\begin{center}
%\begin{footnotesize}
\begin{scriptsize}
\begin{gnuplot}[scale=1,terminal=epslatex,terminaloptions={font 8 color colortext size 15.5cm,9cm}]
set xdata time
set timefmt "%Y-%m-%d"
set format x "%Y\n%b"
set datafile separator ","
set xrange ["2013-01":"2013-11"]
set bmargin 2
unset key
set grid y
set ylabel "Cena otwarcia"
set mxtics 0
set multiplot
set size 1, 0.5
set origin 0, 0.5
plot "GNUwykres/DATA/wig20_d.csv" u 1:2 with impulses  lc rgb "orange" lw 1 lt 1
set origin 0, 0
plot "GNUwykres/DATA/wig20_d.csv" u 1:2:4:3:5 with candlesticks lc rgb "black"
unset multiplot
\end{gnuplot}
\end{scriptsize}
%\end{footnotesize}
\end{center}
\caption{Cena otwarcia indeksu WIG-20 w okresie od 01-01-2012 do 01-10-2013.}
\end{figure}

\begin{lstlisting}
set xdata time
set timefmt "%Y-%m-%d"
set format x "%Y\n%b"
set datafile separator ","
set xrange ["2013-01":"2013-11"]
set bmargin 2
unset key
set grid y
set ylabel "Cena otwarcia"
set mxtics 0
set multiplot
set size 1, 0.5
set origin 0, 0.5
plot "GNUwykres/DATA/wig20_d.csv" u 1:2 with impulses  lc rgb "orange" lw 1 lt 1
set origin 0, 0
plot "GNUwykres/DATA/wig20_d.csv" u 1:2:4:3:5 with candlesticks lc rgb "black"
unset multiplot
\end{lstlisting}

\subsection{Dane doświadczalne}
Gnuplot bardzo często jest wykorzystywany do prezentacji danych doświadczalnych w takich naukach jak: fizyka czy chemia. Poniżej kilka przykładowych wykresów.

\begin{figure}[!ht]
\begin{center}
%\begin{footnotesize}
\begin{scriptsize}
\begin{gnuplot}[scale=1,terminal=epslatex,terminaloptions={font 8 color colortext size 14cm,7.5cm}]
set datafile separator ","
set bmargin 0.5
set format x "%g kg"
set format y "%g $\\mu_3$"
set mxtics 2
set mytics 2
set xtics 2
set xrange[0:21]
set xzeroaxis
p "GNUwykres/DATA/a.csv" u 2:3 t "interpolacja" smooth csplines lw 5 lc rgb "orange" lt 1,"GNUwykres/DATA/a.csv" u 2:($3>=0 ? $3:1/0) t "wynik dodatni" w points pt 7 lc rgb "red","GNUwykres/DATA/a.csv" u 2:($3<0 ? $3:1/0) t "wynik ujemny" w points pt 7 lc rgb "blue","GNUwykres/DATA/a.csv" u 2:($3>0 ? ($3+0.05):1/0 ):3 t "" smooth uniq w labels
\end{gnuplot}
\end{scriptsize}
%\end{footnotesize}
\end{center}
\caption{Interpolacja.}
\end{figure}
\begin{lstlisting}
# Interpolacja
set datafile separator ","
set bmargin 0.5
set format x "%g kg"
set format y "%g $\\mu_3$"
set mxtics 2
set mytics 2
set xtics 2
set xrange[0:21]
set xzeroaxis
p "GNUwykres/DATA/a.csv" u 2:3 t "interpolacja" smooth csplines lw 5 lc rgb "orange" lt 1,"GNUwykres/DATA/a.csv" u 2:($3>=0 ? $3:1/0) t "wynik dodatni" w points pt 7 lc rgb "red","GNUwykres/DATA/a.csv" u 2:($3<0 ? $3:1/0) t "wynik ujemny" w points pt 7 lc rgb "blue","GNUwykres/DATA/a.csv" u 2:($3>0 ? ($3+0.05):1/0 ):3 t "" smooth uniq w labels
\end{lstlisting}

\begin{figure}[!ht]
\begin{center}
%\begin{footnotesize}
\begin{scriptsize}
\begin{gnuplot}[scale=1,terminal=epslatex,terminaloptions={font 8 color colortext size 12.5cm,5.5cm}]
set bmargin 0.5
set datafile separator ","
set key right bottom
set fit errorvariables
set xrange[0:21]
set yrange[0.5:7.5]
set mxtics 5
f(x)=a*x+b
a=1;b=1
fit f(x) "GNUwykres/DATA/b.csv" u 1:2 via a,b
print "error of a is:", a_err
print "error of a is:", b_err
set label "$a$ = %6.4f", a, "$\\quad\\pm$ %6.4f", a_err at 1.25,7.0
set label "$b$ = %6.4f", b, "$\\quad\\pm$ %6.4f", b_err at 1.25,6.5
set label "$s$ = %10.6f",FIT_STDFIT at 1.25,6.0
stat "GNUwykres/DATA/b.csv" u 1:2
set label "$R^2$ = %1.6f",STATS_correlation**2 at 1.25,5.5
#set label "$f(x) =$ a$x+$b" at 1.25,5
p f(x) title "$f(x)=ax+b$" lc rgb "orange" lw 5, "GNUwykres/DATA/b.csv" u 1:2:3  with yerrorbars title "" lt 1 pt 7 lc rgb "black"
\end{gnuplot}
\end{scriptsize}
%\end{footnotesize}
\end{center}
\caption{Regresja liniowa.}
\end{figure}

\begin{lstlisting}
# Regresja liniowa:
set bmargin 0.5
set datafile separator ","
set key right bottom
set fit errorvariables
set xrange[0:21]
set yrange[0.5:7.5]
set mxtics 5
f(x)=a*x+b
a=1;b=1
fit f(x) "GNUwykres/DATA/b.csv" u 1:2 via a,b
print "error of a is:", a_err
print "error of a is:", b_err
set label "$a$ = %6.4f", a, "$\\quad\\pm$ %6.4f", a_err at 1.25,7.0
set label "$b$ = %6.4f", b, "$\\quad\\pm$ %6.4f", b_err at 1.25,6.5
set label "$s$ = %10.6f",FIT_STDFIT at 1.25,6.0
stat "GNUwykres/DATA/b.csv" u 1:2
set label "$R^2$ = %1.6f",STATS_correlation**2 at 1.25,5.5
p f(x) title "$f(x)=ax+b$" lc rgb "orange" lw 5, "GNUwykres/DATA/b.csv" u 1:2:3  with yerrorbars title "" lt 1 pt 7 lc rgb "black"
\end{lstlisting}

\begin{figure}[!ht]
\begin{center}
%\begin{footnotesize}
\begin{scriptsize}
\begin{gnuplot}[scale=1,terminal=epslatex,terminaloptions={font 8 color colortext size 12.5cm,6cm}]
set bmargin 0.5
set key outside
set fit errorvariables
set xrange[0:18.5]
set yrange[-0.5:20.5]
set y2range[-60:2]
set datafile separator ","
set y2tics tc rgb "red"
set ytics nomirror
#set y2label tc rgb "red"
#set format y2 "%2.1e"
#set logscale y2
f(x)=a*b**x
a=1;b=1
fit f(x) "GNUwykres/DATA/bb.csv" u 1:2 via a,b
set label "$f(x)=$ %1.2f",a, "$\\cdot$%1.2f$^x$", b at 12,3
p "GNUwykres/DATA/bb.csv" u 1:(-3*$2) axis x1y2 w points title "dane0" lt 6 lc rgb "red", "GNUwykres/DATA/bb.csv" u 1:2 axis x1y1 w points title "dane1" lt 7 lc rgb "black", f(x) axis x1y1 title "$f(x)$" lc rgb "orange" lw 5 lt 1
\end{gnuplot}
\end{scriptsize}
%\end{footnotesize}
\end{center}
\caption{Regresja nieliniowa.}
\end{figure}
\begin{lstlisting}
# Regresja nieliniowa
set bmargin 0.5
set key outside
set fit errorvariables
set xrange[0:18.5]
set yrange[-0.5:20.5]
set y2range[-60:2]
set datafile separator ","
set y2tics tc rgb "red"
set ytics nomirror
f(x)=a*b**x
a=1;b=1
fit f(x) "GNUwykres/DATA/bb.csv" u 1:2 via a,b
set label "$f(x)=$ %1.2f",a, "$\\cdot$%1.2f$^x$", b at 12,3
p "GNUwykres/DATA/bb.csv" u 1:(-3*$2) axis x1y2 w points title "dane0" lt 6 lc rgb "red", "GNUwykres/DATA/bb.csv" u 1:2 axis x1y1 w points title "dane1" lt 7 lc rgb "black", f(x) axis x1y1 title "$f(x)$" lc rgb "orange" lw 5 lt 1
\end{lstlisting}

\section{Wykresy matematyczne}
\subsection{Funkcje}
Gnuplot bardzo dobrze sprawdza się w rysowaniu funkcji matematycznych. 
\begin{equation*}
f(x) = \begin{cases}
\sin(x) &x<\frac{\pi}{8}\\
\cos\left(x-\frac{\pi}{4}\right) &x\in\left\langle \frac{\pi}{8};2\pi\right\rangle
\end{cases}
\end{equation*}

\begin{figure}[!ht]
\begin{center}
\begin{scriptsize}
\begin{gnuplot}[scale=1,terminal=epslatex,terminaloptions={font 8 color colortext size 15.5cm,3.5cm}]
set xrange [-3*pi/2:2*pi]
set yrange [-1:1.2]
set ytics 1
set xtics 1
set size ratio 0.2000805
set xzeroaxis lt 1 lw 2 lc rgb 'black'
set yzeroaxis lt 1 lw 2 lc rgb 'black'
set border 0
set xtics axis
set ytics axis
set xtics ('$-\frac{3\pi}{2}$' -3*pi/2,'$-\pi$' -pi,\
 '$-\frac{\pi}{2}$' -pi/2, '$\frac{\pi}{2}$' pi/2, '$\pi$' pi,\
 '$\frac{3\pi}{2}$' 3*pi/2, '$2\pi$' 2*pi, '$\frac{\pi}{8}$' pi/8)\
  scale 0.5 offset 0,0.4
set ytics scale 0.5
set format xy '%.0f'
set format x2 '%.2f'
set x2tics scale 1 offset -0.3,-0.4
set x2tics out pi/2
set arrow from -1.5*pi,1.2 to 2*pi,1.2 nohead lt 2 lc 0
set key right top
#set style arrow 1 head filled size screen 0.015,10,35 linewidth 1 lt 2 lc rgb "red"
#set arrow from -2,-1.5 to 3,1 arrowstyle 1
#set style arrow 2 nohead lt 5 lc rgb "green"
#set arrow from -3,-1 to 5,0.5 arrowstyle 2
set bmargin 0
set lmargin 0
set rmargin 0
plot (x<0.4)?sin(x):1/0 t '$f(x)$' lc rgb 'orange' lw 7,\
(x>=0.39 && x<=2*pi)?cos(x-pi/4):1/0 t '' lc rgb 'orange' lw 7 lt 1,\
"<echo '0.39 0.38'" w points pt 7 lc rgb 'orange' ps 1.5 t "",\
"<echo '0.39 0.38'" w points pt 7 lc rgb 'white' ps 1 t "",\
"<echo '0.39 0.92'" w points pt 7 lc rgb 'orange' ps 1.5 t "",\
"<echo '6.28 0.707'" w points pt 7 lc rgb 'orange' ps 1.5 t ""
\end{gnuplot}
\end{scriptsize}
\end{center}
\caption{Wykres funkcji $f(x)$.}
\end{figure}
\begin{lstlisting}
set xrange [-3*pi/2:2*pi]
set yrange [-1:1.2]
set ytics 1
set xtics 1
set size ratio 0.2000805
set xzeroaxis lt 1 lw 2 lc rgb 'black'
set yzeroaxis lt 1 lw 2 lc rgb 'black'
set border 0
set xtics axis
set ytics axis
set xtics ('$-\frac{3\pi}{2}$' -3*pi/2,'$-\pi$' -pi,\
 '$-\frac{\pi}{2}$' -pi/2, '$\frac{\pi}{2}$' pi/2, '$\pi$' pi,\
 '$\frac{3\pi}{2}$' 3*pi/2, '$2\pi$' 2*pi, '$\frac{\pi}{8}$' pi/8)\
  scale 0.5 offset 0,0.4
set ytics scale 0.5
set format xy '%.0f'
set format x2 '%.2f'
set x2tics scale 1 offset -0.3,-0.4
set x2tics out pi/2
set arrow from -1.5*pi,1.2 to 2*pi,1.2 nohead lt 2 lc 0
set key right top
set bmargin 0
set lmargin 0
set rmargin 0
plot (x<0.4)?sin(x):1/0 t '$f(x)$' lc rgb 'orange' lw 7,\
(x>=0.39 && x<=2*pi)?cos(x-pi/4):1/0 t '' lc rgb 'orange' lw 7 lt 1,\
"<echo '0.39 0.38'" w points pt 7 lc rgb 'orange' ps 1.5 t "",\
"<echo '0.39 0.38'" w points pt 7 lc rgb 'white' ps 1 t "",\
"<echo '0.39 0.92'" w points pt 7 lc rgb 'orange' ps 1.5 t "",\
"<echo '6.28 0.707'" w points pt 7 lc rgb 'orange' ps 1.5 t ""
\end{lstlisting}



Kolejny przykład to zaznaczenie dwóch pól $D_1$ i $D_2$ pod wykresem danej funkcji.
\begin{equation*}
D_{1}=\int_0^4\!\! 4\sin\left(\frac{x}{2}\right) dx=4\left[2-2\cos(2)\right]\approx 11.329
\end{equation*}
\begin{equation*}
D_{2}=\int_{-2}^{-4}\!\! 4\sin\left(\frac{x}{2}\right) dx=-8\left[\cos(2)-\cos(1)\right]\approx 7.651
\end{equation*}

\begin{lstlisting}
set xzeroaxis
set key top left
set ylabel "$y$ -- wartości"
set xlabel "$x$ -- argumenty"
set xrange[-7:7]
set yrange[-5:5]
set tics out
set xtics offset 0,0.25
set ytics offset 0.5,0
set size ratio 0.71
plot (x>=0 && x<=4)?4*sin(x/2):1/0 with filledcurve y1=0 lc rgb "FFDAB9" t "$D_{1}=11.329$",(x>=-4 && x<=-2)?4*sin(x/2):1/0 with filledcurve y1=0 lc rgb "F0E68C" t "$D_{2}=7.651$",4*sin(x/2) lc rgb "orange" lw 5 lt 1 t ""
\end{lstlisting}

\begin{figure}[!ht]
\begin{center}
\begin{scriptsize}
\begin{gnuplot}[scale=1,terminal=epslatex,terminaloptions={font 8 color colortext size 12cm,6cm}]
set xzeroaxis
set key top left
#set ylabel "{\\Large $y$ -- wartości}"
set ylabel "$y$ -- wartości"
set xlabel "$x$ -- argumenty"
set xrange[-7:7]
set yrange[-5:5]
set tics out 
set xtics offset 0,0.25
set ytics offset 0.5,0
set size ratio 0.71
plot (x>=0 && x<=4)?4*sin(x/2):1/0 with filledcurve y1=0 lc rgb "#FFDAB9" t "$D_{1}=11.329$",(x>=-4 && x<=-2)?4*sin(x/2):1/0 with filledcurve y1=0 lc rgb "#F0E68C" t "$D_{2}=7.651$",4*sin(x/2) lc rgb "orange" lw 5 lt 1 t ""
\end{gnuplot}
\end{scriptsize}
\end{center}\caption{Całka oznaczona.}
\end{figure}









\subsection{Krzywe}
Kilka przykładów rysowania krzywych opisanych za pomocą równań parametrycznych.
\begin{itemize}
\item[$\bullet$] równanie parametryczne okręgu o postaci $x^2+y^2=1$:
\begin{equation*} 
\begin{cases}
x=\sin t \\
y=\cos t 
\end{cases}
\quad t\in \langle 0, 2\pi )
\end{equation*}
\item[$\bullet$] równanie parametryczne elipsy o postaci $16x^2+y^2=16$:
\begin{equation*} 
\begin{cases}
x=1\frac{1}{2}\sin t \\
y=4\cos t 
\end{cases}
\quad t\in \langle 0, 2\pi )
\end{equation*}
\item[$\bullet$] równanie parametryczne hiperboli o postaci $9x^2-2.25y^2=36$:
\begin{equation*}
\begin{cases}
x=\pm2\cosh t \\
y=\pm4\sinh t 
\end{cases}
\quad t\in \mathcal{R}
\end{equation*}
\item[$\bullet$] równanie parametryczne paraboli o postaci $x=0.1y^2-3$:
\begin{equation*}
\begin{cases}
x= \frac{1}{10}t^2-3\\
y= t 
\end{cases}
\quad t\in \mathcal{R}
\end{equation*}
\item[$\bullet$] równanie parametryczne linii o postaci $x=6$:
\begin{equation*}
\begin{cases}
x= 6\\
y= t 
\end{cases}
\quad t\in \mathcal{R}
\end{equation*}
\item[$\bullet$] równanie parametryczne wielomianu o postaci $y=0.05x^2-4.5$:
\begin{equation*}
\begin{cases}
x= t\\
y= \frac{1}{20}t^2-4\frac{1}{2}
\end{cases}
\quad t\in \mathcal{R}
\end{equation*}
\end{itemize}


\begin{figure}[!ht]
\begin{center}
%\begin{footnotesize}
\begin{scriptsize}
\begin{gnuplot}[scale=1,terminal=epslatex,terminaloptions={font 8 color colortext size 15.5cm,6cm}]
set parametric
set bmargin 0
set grid
set sample 1000
set key outside left center spacing 1.5
set size ratio 0.625
set x2tics ("$\\frac{3}{2}$" 1.5,"$-\\frac{3}{2}$" -1.5,"$-3$" -3,"$3$" 3)
set ytics add ("$-1$" -1,"$1$" 1)
plot [-3*pi:3*pi] [-8:8] [-5:5]\
0+sin(t),0+cos(t) lc rgb "orange" lt 1 lw 5 t "$x^2+y^2=1$",\
0+1.5*sin(t),0+4*cos(t) lc rgb "violet" lt 1 lw 5 t "$16x^2+y^2=16$",\
2*cosh(t),4*sinh(t) lc rgb "red" lt 1 lw 5 t "$9x^2-2.25y^2=36$",\
-2*cosh(t),-4*sinh(t) lc rgb "red" lt 1 lw 5 t "",\
-0.1*t**2-3,t lc rgb "green" lt 1 lw 5 t "$x=-\\frac{1}{10}y^2-3$",\
6,t lc rgb "blue" lt 1 lw 5 t "$x=6$",\
t, 0.05*t**2-4.5 lc rgb "black" lt 1 lw 5 t "$y=\\frac{1}{20}x^2-4\\frac{1}{2}$"
unset parametric
\end{gnuplot}
\end{scriptsize}
%\end{footnotesize}
\end{center}
\caption{Wykresy.}
\end{figure}

\begin{lstlisting}
set parametric
set bmargin 0
set grid
set sample 1000
set key outside left center spacing 1.5
set size ratio 0.625
set x2tics ("$\\frac{3}{2}$" 1.5,"$-\\frac{3}{2}$" -1.5,"$-3$" -3,"$3$" 3)
set ytics add ("$-1$" -1,"$1$" 1)
plot [-3*pi:3*pi] [-8:8] [-5:5]\
sin(t),cos(t) lc rgb "orange" lt 1 lw 5 t "$x^2+y^2=1$",\
1.5*sin(t),4*cos(t) lc rgb "violet" lt 1 lw 5 t "$16x^2+y^2=16$",\
2*cosh(t),4*sinh(t) lc rgb "red" lt 1 lw 5 t "$9x^2-2.25y^2=36$",\
-2*cosh(t),-4*sinh(t) lc rgb "red" lt 1 lw 5 t "",\
-0.1*t**2-3,t lc rgb "green" lt 1 lw 5 t "$x=-\\frac{1}{10}y^2-3$",\
6,t lc rgb "blue" lt 1 lw 5 t "$x=6$",\
t, 0.05*t**2-4.5 lc rgb "black" lt 1 lw 5 t "$y=\\frac{1}{20}x^2-4\\frac{1}{2}$"
unset parametric
\end{lstlisting}



\section*{Uwagi końcowe}
Każdy wykres można zawsze dowolnie formatować modyfikując jego domyślne ustawienia. Poniżej zostanie wypisanych kilka najbardziej przydatnych i najczęściej używanych ustawień tak aby dostosować go do indywidualnych potrzeb.
\begin{lstlisting}
# nazwa osi X:
set xlabel "nazwa"
# zakres osi X:
set xrange[-3:5]
# częstotliwość głównej podziałki na osi X:
set xtics 1
# częstotliwość pomocniczej podziałki na osi X:
set mxtics 4
# skala logarytmiczna na osi X:
set logscale x
# pozycja legendy - z lewej strony, na górze oraz poza wykresem:
set key outside left top
# pozycja tekstu:
set label "tekst" at 1,2 left
# tytuł wykresu:
set title "tytuł"
# siatka wykresu tylko dla osi X:
set grid xtics
# rozmiar górnego marginesu:
set tmargin 0
# kolor osi, podziałki i czcionki osi:
set border 31 lc rgb "blue"
\end{lstlisting}



Wiele bardzo ciekawych materiałów na temat gnuplota można znaleźć pod adresem \href{http://www.gnuplotting.org/}{www.gnuplotting.org}. Także na stronie domowej programu \href{http://www.gnuplot.info/}{www.gnuplot.info} są prezentowane ciekawe przykłady oraz  jego dokumentacja. Zaznaczmy, że z poziomu konsoli gnuplota także mamy dostęp do opisu wybranej funkcji, wystarczy wpisać komendę {\verb help } a następnie parametr jaki nas interesuje.
\begin{lstlisting}
# opis funkcji fitowania - dopasowywanie krzywej:
help fit
\end{lstlisting}






\end{document}
